\chapter{Variabili non Discrezionali} \label{CAP:cinque}
\bigskip
Iniziamo questo capitolo introducendo alcune definizioni.
\begin{definiz}
La piena efficienza (BCC o CCR) \'e raggiunta da una $DMU_o$ se e solo se sono soddisfatte le seguenti condizioni:
\begin{equation}
\begin{split}
\text{(i) } \qquad &  \theta^* = 1 \\
\text{(ii)} \qquad & \text{tutti gli slack sono nulli}\\
\end{split}
\end{equation}
\end{definiz}
\section{NCN}
\begin{definiz} Non-Controllable Variable Model (NCN):
\begin{equation}
\begin{split}
\text{(NCN)} \qquad & \min_{\theta, \boldsymbol{\lambda}} \theta \\
\text{t.c.} \qquad & \theta \boldsymbol{x^C_o} \geq X^C\boldsymbol{\lambda} \\
& \boldsymbol{y^C_o} \leq Y^C\boldsymbol{\lambda} \\
& \boldsymbol{x^N_o} = X^N\boldsymbol{\lambda} \\
& \boldsymbol{y^N_o} = Y^N\boldsymbol{\lambda} \\
& L \leq \boldsymbol{e\lambda} \leq U \\
& \boldsymbol{\lambda} \geq \boldsymbol{0} \\
\end{split}
\end{equation}
\end{definiz}
\section{NDSC}
\begin{definiz} Non-discretionary Variable Model (NDSC):
\begin{equation}
\begin{split}
\text{(NDSC)} \qquad & \min \theta - \epsilon(\Sigma_{i \in D} s^-_i + \Sigma^s_{r=1} s^+_r) \\
\text{t.c} \qquad &  \theta x_{io} = \Sigma^n_{j=1}x_ij\lambda_j + s^-_i, \qquad i \in D \\
& x_{io} = \Sigma^n_{j=1}x_ij\lambda_j + s^-_i, \qquad i \in ND \\
& y_{ro} = \Sigma^n_{j=1}y_rj\lambda_j - s^+_r, \qquad r = 1,\dots, s \\
\end{split}
\end{equation}
\end{definiz}
(TODO eventualmente scriverne il duale)
\section{BND}
\begin{definiz} Bounded Variable Model (BND):\\\\
(a)Input Oriented Bounded Variable Model
\begin{equation}
\begin{split}
(BND_o) \qquad & \min \theta \\
\text{t.c} \qquad &  \theta \boldsymbol{x_{o}^C} \geq X^C\boldsymbol{\lambda} \\
&  \boldsymbol{y_{o}^C} \leq Y^C\boldsymbol{\lambda} \\
&  \boldsymbol{l_{o}^{N_{x}}} \leq X^N \boldsymbol{\lambda} \leq \boldsymbol{u_{o}^{N_{x}}} \\
&  \boldsymbol{l_{o}^{N_{y}}} \leq Y^N \boldsymbol{\lambda} \leq \boldsymbol{u_{o}^{N_{y}}} \\
&  L \leq \boldsymbol{e \lambda} \leq U \\
&  \boldsymbol{\lambda \geq 0}   \\
\end{split}
\end{equation}
(b)Output Oriented Bounded Variable Model
\begin{equation}
\begin{split}
(BNDO_o) \qquad & \max \eta \\
\text{t.c} \qquad &  \boldsymbol{x_{o}^C} \geq X^C\boldsymbol{\lambda} \\
&  \eta \boldsymbol{y_{o}^C} \leq Y^C\boldsymbol{\lambda} \\
&  \boldsymbol{l_{o}^{N_{x}}} \leq X^N \boldsymbol{\lambda} \leq \boldsymbol{u_{o}^{N_{x}}} \\
&  \boldsymbol{l_{o}^{N_{y}}} \leq Y^N \boldsymbol{\lambda} \leq \boldsymbol{u_{o}^{N_{y}}} \\
&  L \leq \boldsymbol{e \lambda} \leq U \\
&  \boldsymbol{\lambda \geq 0}   \\
\end{split}
\end{equation}
dove $(\boldsymbol{l_{o}^{N_{x}}, u_{o}^{N_{x}}})$ e $(\boldsymbol{l_{o}^{N_{y}}, u_{o}^{N_{y}}})$ sono i vettori del lower e upper bound per gli input e output non discrezionali della $DMU_o$. Infine $(\boldsymbol{x_o^N, y_o^N})$ non sono inclusi nella formula perch\'e si suppone che si trovino tra i due limiti precedentemente espressi.
\end{definiz}