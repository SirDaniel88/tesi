\chapter{Ritorno di Scala} \label{CAP:tre}
\section{Ritorno di scala}
\bigskip

TODO: scrivere l'introduzione
Ora consideriamo le condizioni per i ritorni di scala dei seguenti modelli

\section{Il ritorno di scale del BBC-Model}

Consideriamo le equazioni del BCC model:
\begin{equation}
\begin{split}
\max \qquad & z = \boldsymbol{uy_o} - u_o \\
\text{t.c} \qquad & \boldsymbol{vx_o = 1} \\
& \boldsymbol{-vX + uY -} u_o\boldsymbol{e \leq 0} \\
& \boldsymbol{v \geq 0, u \geq 0,} u_o \text{ qualunque} \\ 
\end{split}
\end{equation}
possiamo esprime i ritorni di scala per tale modello utilizzando il seguente teorema.

\begin{teor} Assumiamo che $(\boldsymbol{x_o, y_o})$ sia una punto della frontiera dell'efficienza allora abbiamo che: \\
(i)L'Increasing returns-to-scale prevale su $(\boldsymbol{x_o, y_o})$ se e solo se $u^*_o < 0$ per ogni soluzioni ottimale. \\
(ii)Il Decreasing returns-to-scale prevale su $(\boldsymbol{x_o, y_o})$ se e solo se $u^*_o > 0$ per ogni soluzioni ottimale.\\
(ii)Il Constant returns-to-scale prevale su $(\boldsymbol{x_o, y_o})$ se e solo se $u^*_o = 0$ in alcune soluzioni ottimali.\\
\end{teor}

\section{Il ritorno di scale del CCR-Model}

\begin{teor}
Sia $(\boldsymbol{x_o, y_o})$ sia una punto della frontiera dell'efficienza, e consideriamo la soluzione ottima ottenuta dal CCR-Model $(\lambda^*_1, \dots, \lambda^*_n)$. Il ritorno di scale in tale punto può essere determinato dalle seguenti condizioni:\\
(i) Se $\Sigma^n_{j = 1} \lambda^*_j = 1$ in ciascuna soluzione ottimale allora prevale il constant returns-to-scale \\
(ii) Se $\Sigma^n_{j = 1} \lambda^*_j > 1$ in ciascuna soluzione ottimale allora prevale il decreasing returns-to-scale \\
(iii) Se $\Sigma^n_{j = 1} \lambda^*_j < 1$ in ciascuna soluzione ottimale allora prevale l'increasing returns-to-scale \\
\end{teor}

Enunciamo adesso un teorema che mette in relazione il BBC-Model e CCR-Model.

\begin{teor}
Consideriamo le soluzioni ottimali del CCR-Model e del BCC-Model si ha che: \\
(i) $u^*_o > 0$ per ogni soluzione ottimale del BBC-Model se e solo se $\Sigma^n_{j = 1} \hat{\lambda}^*_j > 1$ per ogni soluzione ottimale del CCR-Model corrispondente.\\
(ii) $u^*_o < 0$ per ogni soluzione ottimale del BBC-Model se e solo se $\Sigma^n_{j = 1} \hat{\lambda}^*_j < 1$ per ogni soluzione ottimale del CCR-Model corrispondente.\\
(iii) $u^*_o = 0$ per alcune soluzione ottimali del BBC-Model se e solo se $\Sigma^n_{j = 1} \hat{\lambda}^*_j = 1$ per alcune soluzioni ottimali del CCR-Model corrispondente.\\
\end{teor}

\section{Grandezze di scala produttive}

\begin{teor} Una $DMU_o$ efficiente per il CCR-Model sarà efficiente anche per il BCC-Model model e prevale il constant returns-to-scale.
\end{teor}
TODO scrivere def MPSS
\begin{teor} Consideriamo una $DMU_o$ con input e output rappresentati dai vettori $\boldsymbol{x_o, y_o}$. Si ha che una condizione necessaria per essere MPSS \'e  $\beta^*/\alpha^* = \max \beta/\alpha = 1$. In tale caso $\beta^* = \alpha^*$ e i ritorni di scala sono costanti. 
\end{teor}
