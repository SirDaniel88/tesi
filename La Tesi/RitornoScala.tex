\chapter{Ritorno di Scala} \label{CAP:tre}
\section{Ritorno di scala}
\bigskip

TODO: scrivere l'introduzione
Ora consideriamo le condizioni per i ritorni di scala dei seguenti modelli

\section{Il ritorno di scale del BBC-Model}

Consideriamo le equazioni del BCC model:
\begin{equation}
\begin{split}
\max \qquad & z = \boldsymbol{uy_o} - u_o \\
\text{t.c} \qquad & \boldsymbol{vx_o = 1} \\
& \boldsymbol{-vX + uY -} u_o\boldsymbol{e \leq 0} \\
& \boldsymbol{v \geq 0, u \geq 0,} u_o \text{ qualunque} \\ 
\end{split}
\end{equation}
possiamo esprime i ritorni di scala per tale modello utilizzando il seguente teorema.

\begin{teor} Assumiamo che $(\boldsymbol{x_o, y_o})$ sia una punto della frontiera dell'efficienza allora abbiamo che: \\
(i)L'Increasing returns-to-scale prevale su $(\boldsymbol{x_o, y_o})$ se e solo se $u^*_o < 0$ per ogni soluzioni ottimale. \\
(ii)Il Decreasing returns-to-scale prevale su $(\boldsymbol{x_o, y_o})$ se e solo se $u^*_o > 0$ per ogni soluzioni ottimale.\\
(ii)Il Constant returns-to-scale prevale su $(\boldsymbol{x_o, y_o})$ se e solo se $u^*_o = 0$ in alcune soluzioni ottimali.\\
\end{teor}

\section{Il ritorno di scale del CCR-Model}

\begin{teor}
Sia $(\boldsymbol{x_o, y_o})$ sia una punto della frontiera dell'efficienza, e consideriamo la soluzione ottima ottenuta dal CCR-Model $(\lambda^*_1, \dots, \lambda^*_n)$. Il ritorno di scale in tale punto può essere determinato dalle seguenti condizioni:\\
(i) Se $\Sigma^n_{j = 1} \lambda^*_j = 1$ in ciascuna soluzione ottimale allora prevale il constant returns-to-scale \\
(ii) Se $\Sigma^n_{j = 1} \lambda^*_j > 1$ in ciascuna soluzione ottimale allora prevale il decreasing returns-to-scale \\
(iii) Se $\Sigma^n_{j = 1} \lambda^*_j < 1$ in ciascuna soluzione ottimale allora prevale l'increasing returns-to-scale \\
\end{teor}

Enunciamo adesso un teorema che mette in relazione il BBC-Model e CCR-Model.

\begin{teor}
Consideriamo le soluzioni ottimali del CCR-Model e del BCC-Model si ha che: \\
(i) $u^*_o > 0$ per ogni soluzione ottimale del BBC-Model se e solo se $\Sigma^n_{j = 1} \hat{\lambda}^*_j > 1$ per ogni soluzione ottimale del CCR-Model corrispondente.\\
(ii) $u^*_o < 0$ per ogni soluzione ottimale del BBC-Model se e solo se $\Sigma^n_{j = 1} \hat{\lambda}^*_j < 1$ per ogni soluzione ottimale del CCR-Model corrispondente.\\
(iii) $u^*_o = 0$ per alcune soluzione ottimali del BBC-Model se e solo se $\Sigma^n_{j = 1} \hat{\lambda}^*_j = 1$ per alcune soluzioni ottimali del CCR-Model corrispondente.\\
\end{teor}

\section{Grandezze di scala produttive}

\begin{teor} Una $DMU_o$ efficiente per il CCR-Model sar\'a efficiente anche per il BCC-Model model e prevar\'a il constant returns-to-scale.
\end{teor}
TODO scrivere def MPSS
\begin{definiz}Una $DMU_o$ si dice MPSS se soddisfa le seguenti condizioni:
\begin{equation}
\begin{split}
\text{(i)} \qquad & \beta^* / \alpha^* \\
\text{(ii)} \qquad & \text{tutti gli slack sono zero}\\
\end{split}
\end{equation}
\end{definiz}
\begin{teor} Consideriamo una $DMU_o$ con input e output rappresentati dai vettori $\boldsymbol{x_o, y_o}$. Si ha che una condizione necessaria per essere MPSS \'e  $\beta^*/\alpha^* = \max \beta/\alpha = 1$. In tale caso $\beta^* = \alpha^*$ e i ritorni di scala sono costanti. 
\end{teor}
\begin{teor} Nel BCC model un insieme di riferimento per ciascuna $DMU$ non efficiente non può includere allo stesso tempo increasing e decreasing return-to-scale DMUs.
\end{teor}
\begin{cor} \label{cor:return-to-scale} 
Sia $E_o$ l'insieme di riferimento di una $DMU(\boldsymbol{x_o, y_0})$. Allora, $E_o4$ sar\'a costituita da una delle seguenti combinazioni di $DMU$ efficienti:
\begin{equation}
\begin{split}
\text{(i)} \qquad & \text{Tutte le DMU sono IRS} \\
\text{(ii)} \qquad & \text{Tutte le DMU sono CRS} \\
\text{(iii)} \qquad & \text{Tutte le DMU sono DRD} \\
\text{(iv)} \qquad & \text{le DMU sono IRS o CRS} \\
\text{(v)} \qquad & \text{le DMU sono DRS o CRS} \\
\end{split}
\end{equation}
dove IRS, CRS e DRS sta per increasing, constant e decresing return-to-scale rispettivamente.
\end{cor}
\begin{teor} \label{EQ:return-to-scale}
Sia $(\hat{\boldsymbol{x_o}}, \hat{\boldsymbol{y_o}})$ la proiezione efficiente di una $DMU_o$ che risulti BCC-inefficient, sia $E_o$ l'insieme di riferimento relativo alla $DMU_o$. Allora si ha che:\\\\
1. $(\boldsymbol{\hat{x_o}}, \hat{\boldsymbol{y_o}})$ \'e IRS se $E_o$ /'e costituito da $DMUs$ del tipo (i) o (iv) del Corollario \ref{cor:return-to-scale} \\\\
2. $(\boldsymbol{\hat{x_o}}, \hat{\boldsymbol{y_o}})$ \'e DRS se $E_o$ /'e costituito da $DMUs$ del tipo (iii) o (v) del Corollario \ref{cor:return-to-scale} \\\\
\end{teor}
\section{Rilassamento della condizione di convessit\'a}
\bigskip
Possiamo estendere il BCC model rilassando la condizione di convessit\'a $\boldsymbol{e\lambda} = 1$ utilizzando:
\begin{equation}
L \leq \boldsymbol{ e\lambda} \leq U
\end{equation}
dove $L (0 \leq L \leq 1)$ e $U (1 \leq U)$ sono rispettivamente upper e lower bound per la somma dei $\lambda_j$. Notiamo che la condizione $L = 0$ e $U = \infty$ corrisponde al CCR model.
\subsection{Increasing Returns-to-Scale (IRS) Model}
\bigskip
Il caso $L = 1, U = \infty$ \'e chiamato IRS o NDRS (Non decreasing Returns-to-Scale) model. In questo modello il vincolo imposto ai valori di $\boldsymbol{\lambda}$ \'e:
\begin{equation}
\boldsymbol{e\lambda \geq 1}
\end{equation}
Osserviamo che la condizione $L=1$ corrisponde al fatto che non sarà possibile ridurre i pesi della $DMU$ ma sar\'a possibile espandere quest'ultimi all'infinito (TODO inserire immagine con esempio).
\subsection{Decreasing Returns-to-Scale (DRS) Model}
\bigskip
Il caso $L = 0, U = 1$ \'e chiamato DRS o NIRS (Non increasing Returns-to-Scale) model. In questo modello il vincolo imposto ai valori di $\boldsymbol{\lambda}$ \'e:
\begin{equation}
0 \leq \boldsymbol{e\lambda} \leq 1
\end{equation}
Osserviamo che la condizione $U=1$ corrisponde al fatto che non sarà possibile aumentare i pesi della $DMU$ (TODO inserire immagine con esempio).
